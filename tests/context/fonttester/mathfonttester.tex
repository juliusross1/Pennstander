% !TEX root = mathfonttester.tex
\usetypescriptfile[type-imp-pennstander-local]



\setupmathematics[default=normal,lcgreek=normal,ucgreek=normal]

\definemathematics[normaldefaults][default=normal,lcgreek=none,ucgreek=none,mathstyle=scriptscript]

\input mathfonttester_axes.tex
 
\definepapersize[sheet][width=20cm]
\setuppapersize[sheet][sheet]
\setuplayout[width=14cm,rightmargin=3cm,leftmargin=3cm]


\starttext


\define[1]\MyTinyStyle{\switchtobodyfont[6pt]\color[gray]{#1}}
\setupmargindata
  [style=\MyTinyStyle]

\setupinteraction[state=start]
\setupindenting[no]
\completecontent 

\def\allfonts{pennstander-thin,pennstander-extralight,pennstander-light,pennstander-regular,pennstander-medium,pennstander-semibold,pennstander-bold,pennstander-extrabold,pennstander-black}
\def\somefonts{pennstander-thin,pennstander-regular,pennstander-black}
\def\manyfonts{pennstander-thin,pennstander-light,pennstander-regular,pennstander-bold,pennstander-black}

\def\thin{pennstander-thin}


\let\fontlist\somefonts
\let\alphabetlist\alphabetlisttext

\let\stylelist\allstyles
%\def\stylelist{text}


%\math{\scripttext{\fenced[bracket][size=4]{H}}}


\def\sizelist{.2cm,.4cm,1cm} %used for extended shapes such as integrals
%\def\newstyles{normaltext} %uncomment to supress testing of superscripts for various tests


\startbuffer[fonttest]

%\def\sizelist{1cm}
% \showalphabets
 \showintegrals
% \showfencesI
% \showfencesII
% \showfencesIII
%showbraces
% \showradicals
% \showwideaccents
% \showundernestedbraces
% \showovernestedbraces
% \showarrows
% \showprimes
% \showfractions

% \def\stylelist{text}
% \def\newstyles{normaltext}
% \showaccents
% \showspacing
% \showsuperscriptsI
% \showsuperscriptsII
% \showsymbols
\stopbuffer


\definelayer[MyLayer][
  x=0mm, y=0mm,
  width=\paperwidth, height=\paperheight
]


% \section{Integrals}
% \def\sizelist{.2cm,.4cm,1cm}
% \doloopoverlistTHREE{\allintegrals}{\integral}{\crlf}{\sizelist}{\size}{\crlf}{\stylelist}{\style}{\crlf}
% {\math[mathstyle=\style]{\integral[size=\size,bottom=a,top=b]{A}}}



% \math{H}
% \math[mathstyle=script]{H}
% \math[mathstyle=scriptscript]{H}

% \math{\overparent{H}}
% \math[mathstyle=script]{\overparent{H}}
% \math[mathstyle=scriptscript]{\overparent{H}}
% \math{H 2^H 2^{2^H}}
% \math{H 2^{\overparent{H}} 2^{2^{\overparent{H}}}}
% \math{H 2^{\hat{H}} 2^{2^{\hat{H}}}}


\doloopoverfonts{\fontlist}{
    \getbuffer[fonttest]
}

%\math[mathstyle=text]{\overbrace{HH}}
%\math[mathstyle=script]{\overbrace{HH}}
%\math[mathstyle=scriptscript]{\overbrace{HH}}

%\math{2^{\overbrace{HH}}}
%\math{2^{2^{\overbrace{HH}}}}




\stoptext




